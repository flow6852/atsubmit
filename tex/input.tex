×

\paragraph{Contest started}

AtCoder Beginner Contest 153 has begun.

Close

×

\paragraph{Contest is over}

AtCoder Beginner Contest 153 has ended.

Close

\href{/home}{}

\begin{itemize}
\tightlist
\item
  \href{/contests/abc153}{AtCoder Beginner Contest 153}
\end{itemize}

\begin{itemize}
\tightlist
\item
  \protect\hyperlink{}{\includegraphics{//img.atcoder.jp/assets/top/img/flag-lang/en.png}
    English}

  \begin{itemize}
  \tightlist
  \item
    \href{/contests/abc153/tasks/abc153_a?lang=ja}{\includegraphics{//img.atcoder.jp/assets/top/img/flag-lang/ja.png}
    日本語}
  \item
    \href{/contests/abc153/tasks/abc153_a?lang=en}{\includegraphics{//img.atcoder.jp/assets/top/img/flag-lang/en.png}
    English}
  \end{itemize}
\item
  \protect\hyperlink{}{flow\_6852 (Guest)}

  \begin{itemize}
  \item
    \href{/users/flow_6852}{My Profile}
  \item
  \item
    \href{/settings}{General Settings}
  \item
    \href{/settings/icon}{Change Photo}
  \item
    \href{/settings/password}{Change Password}
  \item
    \href{/settings/fav}{Manage Fav}
  \item
  \item
    \href{javascript:void(form_logout.submit())}{Sign Out}
  \end{itemize}
\end{itemize}

{Contest Duration:
\href{http://www.timeanddate.com/worldclock/fixedtime.html?iso=20200126T2100\&p1=248}{2020-01-26
21:00:00+0900} \textasciitilde{}
\href{http://www.timeanddate.com/worldclock/fixedtime.html?iso=20200126T2240\&p1=248}{2020-01-26
22:40:00+0900} (local time) (100 minutes)} {\href{/home}{Back to Home}}

\begin{itemize}
\tightlist
\item
  \href{/contests/abc153}{Top}
\item
  \href{/contests/abc153/tasks}{Tasks}
\item
  \href{/contests/abc153/clarifications}{Clarifications}
\item
  \href{/contests/abc153/submit?taskScreenName=abc153_a}{Submit}
\item
  \protect\hyperlink{}{Results}

  \begin{itemize}
  \item
    \href{/contests/abc153/submissions}{All Submissions}
  \item
    \href{/contests/abc153/submissions/me}{My Submissions}
  \item
  \item
    \href{/contests/abc153/score}{My Score}
  \end{itemize}
\item
  \href{/contests/abc153/standings}{Standings}
\item
  \href{/contests/abc153/standings/virtual}{Virtual Standings}
\item
  \href{/contests/abc153/custom_test}{Custom Test}
\item
  \href{https://img.atcoder.jp/abc153/editorial.pdf}{Editorial}
\item
  \href{https://codeforces.com/blog/entry/73361}{Discuss}
\item
  \href{javascript:void(0)}{}
\end{itemize}

A - Serval vs Monster
\includegraphics{//img.atcoder.jp/assets/top/img/flag-lang/ja.png} /
\includegraphics{//img.atcoder.jp/assets/top/img/flag-lang/en.png}

\begin{center}\rule{0.5\linewidth}{\linethickness}\end{center}

Time Limit: 2 sec / Memory Limit: 1024 MB

配点 : 100 点

\subsubsection{問題文}

サーバルはモンスターと戦っています。

モンスターの体力は H です。

サーバルが攻撃を 1 回行うとモンスターの体力を A 減らすことができます。
攻撃以外の方法でモンスターの体力を減らすことはできません。

モンスターの体力を 0 以下にすればサーバルの勝ちです。

サーバルがモンスターに勝つために必要な攻撃の回数を求めてください。

\subsubsection{制約}

\begin{itemize}
\tightlist
\item
  1 \textbackslash leq H \textbackslash leq 10\^{}4
\item
  1 \textbackslash leq A \textbackslash leq 10\^{}4
\item
  入力中のすべての値は整数である。
\end{itemize}

\begin{center}\rule{0.5\linewidth}{\linethickness}\end{center}

\subsubsection{入力}

入力は以下の形式で標準入力から与えられる。

\begin{verbatim}
H A
\end{verbatim}

\subsubsection{出力}

サーバルがモンスターに勝つために必要な攻撃の回数を出力せよ。

\begin{center}\rule{0.5\linewidth}{\linethickness}\end{center}

\subsubsection{入力例 1}

\begin{verbatim}
10 4
\end{verbatim}

\subsubsection{出力例 1}

\begin{verbatim}
3
\end{verbatim}

\begin{itemize}
\tightlist
\item
  1 回目の攻撃の後のモンスターの体力は 6 です。
\item
  2 回目の攻撃の後のモンスターの体力は 2 です。
\item
  3 回目の攻撃の後のモンスターの体力は -2 です。
\end{itemize}

よって 3 回の攻撃でモンスターに勝つことができます。

\begin{center}\rule{0.5\linewidth}{\linethickness}\end{center}

\subsubsection{入力例 2}

\begin{verbatim}
1 10000
\end{verbatim}

\subsubsection{出力例 2}

\begin{verbatim}
1
\end{verbatim}

\begin{center}\rule{0.5\linewidth}{\linethickness}\end{center}

\subsubsection{入力例 3}

\begin{verbatim}
10000 1
\end{verbatim}

\subsubsection{出力例 3}

\begin{verbatim}
10000
\end{verbatim}

Score : 100 points

\subsubsection{Problem Statement}

Serval is fighting with a monster.

The \emph{health} of the monster is H.

In one attack, Serval can decrease the monster's health by A. There is
no other way to decrease the monster's health.

Serval wins when the monster's health becomes 0 or below.

Find the number of attacks Serval needs to make before winning.

\subsubsection{Constraints}

\begin{itemize}
\tightlist
\item
  1 \textbackslash leq H \textbackslash leq 10\^{}4
\item
  1 \textbackslash leq A \textbackslash leq 10\^{}4
\item
  All values in input are integers.
\end{itemize}

\begin{center}\rule{0.5\linewidth}{\linethickness}\end{center}

\subsubsection{Input}

Input is given from Standard Input in the following format:

\begin{verbatim}
H A
\end{verbatim}

\subsubsection{Output}

Print the number of attacks Serval needs to make before winning.

\begin{center}\rule{0.5\linewidth}{\linethickness}\end{center}

\subsubsection{Sample Input 1}

\begin{verbatim}
10 4
\end{verbatim}

\subsubsection{Sample Output 1}

\begin{verbatim}
3
\end{verbatim}

\begin{itemize}
\tightlist
\item
  After one attack, the monster's health will be 6.
\item
  After two attacks, the monster's health will be 2.
\item
  After three attacks, the monster's health will be -2.
\end{itemize}

Thus, Serval needs to make three attacks to win.

\begin{center}\rule{0.5\linewidth}{\linethickness}\end{center}

\subsubsection{Sample Input 2}

\begin{verbatim}
1 10000
\end{verbatim}

\subsubsection{Sample Output 2}

\begin{verbatim}
1
\end{verbatim}

\begin{center}\rule{0.5\linewidth}{\linethickness}\end{center}

\subsubsection{Sample Input 3}

\begin{verbatim}
10000 1
\end{verbatim}

\subsubsection{Sample Output 3}

\begin{verbatim}
10000
\end{verbatim}

\begin{center}\rule{0.5\linewidth}{\linethickness}\end{center}

Language

C++14 (GCC 5.4.1) Bash (GNU bash v4.3.11) C (GCC 5.4.1) C (Clang 3.8.0)
C++14 (Clang 3.8.0) C\# (Mono 4.6.2.0) Clojure (1.8.0) Common Lisp (SBCL
1.1.14) D (DMD64 v2.070.1) D (LDC 0.17.0) D (GDC 4.9.4) Fortran
(gfortran v4.8.4) Go (1.6) Haskell (GHC 7.10.3) Java7 (OpenJDK 1.7.0)
Java8 (OpenJDK 1.8.0) JavaScript (node.js v5.12) OCaml (4.02.3) Pascal
(FPC 2.6.2) Perl (v5.18.2) PHP (5.6.30) Python2 (2.7.6) Python3 (3.4.3)
Ruby (2.3.3) Scala (2.11.7) Scheme (Gauche 0.9.3.3) Text (cat) Visual
Basic (Mono 4.0.1) C++ (GCC 5.4.1) C++ (Clang 3.8.0) Objective-C (GCC
5.3.0) Objective-C (Clang3.8.0) Swift (swift-2.2-RELEASE) Rust (1.15.1)
Sed (GNU sed 4.2.2) Awk (mawk 1.3.3) Brainfuck (bf 20041219) Standard ML
(MLton 20100608) PyPy2 (5.6.0) PyPy3 (2.4.0) Crystal (0.20.5) F\# (Mono
4.0) Unlambda (0.1.3) Lua (5.3.2) LuaJIT (2.0.4) MoonScript (0.5.0)
Ceylon (1.2.1) Julia (0.5.0) Octave (4.0.2) Nim (0.13.0) TypeScript
(2.1.6) Perl6 (rakudo-star 2016.01) Kotlin (1.0.0) PHP7 (7.0.15) COBOL -
Fixed (OpenCOBOL 1.1.0) COBOL - Free (OpenCOBOL 1.1.0)

Source Code

※ at most 512 KiB\\
※ Your source code will be saved as Main.\emph{extension}.

~ Open File

Toggle Editor

Auto Height

Submit

\begin{center}\rule{0.5\linewidth}{\linethickness}\end{center}

{} {} {} \href{https://www.addtoany.com/share}{}

\begin{center}\rule{0.5\linewidth}{\linethickness}\end{center}

\begin{itemize}
\tightlist
\item
  \href{/contests/abc153/rules}{Rule}
\item
  \href{/contests/abc153/glossary}{Glossary}
\end{itemize}

\begin{itemize}
\tightlist
\item
  \href{/tos}{Terms of service}
\item
  \href{/privacy}{Privacy Policy}
\item
  \href{/personal}{Information Protection Policy}
\item
  \href{/company}{Company}
\item
  \href{/faq}{FAQ}
\item
  \href{/contact}{Contact}
\end{itemize}

{Copyright Since 2012 ©\href{http://atcoder.co.jp}{AtCoder Inc.} All
rights reserved.}

Page Top
